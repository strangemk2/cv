%xelatex jp.tex
\documentclass[10pt,a4paper,sans]{moderncv}

% moderncv 主题
\moderncvstyle{classic}
\moderncvcolor{blue}
%\nopagenumbers{}

\usepackage{fontspec}
\usepackage{xunicode}
\usepackage{xeCJK}
%\setmainfont{Tahoma}
%\setsansfont{Consolas}
%\setmonofont{Courier New}
\setCJKmainfont{Meiryo}
\setCJKsansfont{MS Gothic}
\setCJKmonofont{MS Mincho}

\usepackage{setspace}

\usepackage[scale=0.75]{geometry}
\recomputelengths
\recomputecvlengths
\setlength{\hintscolumnwidth}{2.7cm}

% 个人信息
\firstname{}
\familyname{}
\title{履歴書}
%\address{Room 603, 3-22-11, Ryoukoku, Sumida-ku}{Tokyo, Japan 〒130-0026}
%\mobile{(+81) 080-8155-6536}
\mobile{(+86) 136-1194-1185}
%\phone{(+86) 021-65668510}
%\fax{+3~(456)~789~012}
\email{void@v2mail.net}
%\homepage{www.pm525.com}
%\extrainfo{附加信息 (可选项)}
%\photo[64pt][0.4pt]{picture}   % ‘64pt’是图片必须压缩至的高度、‘0.4pt‘是图片边框的宽度 (如不需要可调节至0pt)、’picture‘ 是图片文件的名字;可选项、如不需要可删除本行
\quote{概要\newline{}
\small
\begin{itemize}
\item 12年渡りソフトウェア開発について全般的な経験
\item トリリンガルで、中国語、日本語、英語を話せる
\item 10人チームの管理経験
\item C++, Perl, Linux/Unix得意
\item 大型E-mailシステムについて豊富な知識
\item ゲーム業界の経験
%\item Very well team work spirit.
\end{itemize}
\normalsize
}

\begin{document}
\begin{spacing}{1.2}
\maketitle

\section{個人情報}
\cvline{名前}{奚\ 雋杰}
\cvline{カタカナ}{\small ケイ\ シュンケツ\normalsize}
\cvline{誕生日}{1982-08-28 (上海)}
\cvline{国籍}{中国}
%\cvline{LinkedIn}{\small \url{http://cn.linkedin.com/in/xijuanjie}\normalsize}
%\cvline{Github}{\small \url{https://github.com/strangemk2}\normalsize}
%\cvline{Blog}{\small \url{<Link to profile>}\normalsize}
%\cvline{Skype}{\small <skype>\normalsize}

\section{経歴}
\cventry{2011/01 - 今}{Tech リーダー}{ヒューレット・パッカード}{上海/東京}{}{}
\cventry{2007/10 - 2010/12}{プログラマー}{フロム・ソフトウェア}{東京}{日本}{}
\cventry{2004/07 - 2007/08}{ソフトウェアエンジニア}{啓明ソフトウェア}{上海}{中国}{}

\section{技術}
\cvcomputer{言語}{C++, Perl, Python, php}{DB}{sqlite, mysql}
\cvcomputer{プラットホーム}{Debian, Gentoo, Redhat, HP/UX, OS X, Windows, postfix, dovecot, apache, nginx}{ツール}{vim, eclipse, clang, git, Mercurial, subversion, trac, redmine, *nix binutils}

\section{教育}
\cventry{2000 - 2004}{建築環境と設備工程}{学士}{上海理工大学}{上海}{}  % 第3到第6编码可留白
\cventry{1996 - 2000}{上海建設中学}{}{}{上海}{}  % 第3到第6编码可留白

%\cventry{年 -- 年}{学位}{院校}{城市}{\textit{成绩}}{说明}

%\section{毕业论文}
%\cvitem{题目}{\emph{题目}}
%\cvitem{导师}{导师}
%\cvitem{说明}{\small 论文简介}

\clearpage

%Section
%\section{言語}
%
%\hspace{25mm}\small Self-assessment level from A to E (A maximum evaluation)\normalsize
%\vspace{5mm}
%
%\begin{tabular}{p{67mm} p{40mm} p{40mm} p{20mm}}
%& \textbf{Understanding} & \textbf{Speaking} & \textbf{Writing} \\
%\end{tabular}
%
%\begin{tabular}{p{67mm} p{20mm} p{20mm} p{20mm} p{20mm} p{20mm}}
%& Listening & Reading & Interaction & Production & \\
%\end{tabular}
%
%\vspace{3mm}
%%lvl should be in this range A1 < A2 < B1 < B2 < C1 < C2
%\cvlanguage{Mandarin}{native}{
%	\begin{tabular}{p{20mm} p{20mm} p{20mm} p{20mm} p{21mm}}
%		A & A & A & A & A
%	\end{tabular}}
%\cvlanguage{Japanese}{bilingual}{
%	\begin{tabular}{p{20mm} p{20mm} p{20mm} p{20mm} p{21mm}}
%		A & A & A & B & B
%	\end{tabular}}
%\cvlanguage{English}{working proficiency}{
%	\begin{tabular}{p{20mm} p{20mm} p{20mm} p{20mm} p{21mm}}
%		B & B & C & C & B
%	\end{tabular}}

\section{認証}
\cvitemwithcomment{LPIC Level1}{}{2009/12}
\cvitemwithcomment{LPIC Level2}{}{2013/05}
\cvitemwithcomment{上級プログラマー}{(日本の応用情報技術者試験と相互承認)}{2006/10}
\cvitemwithcomment{JLPT 1}{}{2007/08}

\section{試験}
\cvline{TOEIC}{840}

\section{プロジェクト}
\subsection{@ヒューレットパッカード}
\cventry{2011/01 - 今}{大規模E-mailシステムのメンテナンス}{}{}{}{
このプロジェクトは日本の大手通信会社の大規模E-mailシステムのメンテナンスである。\\
メンテナンス自体は新しい機能の開発、サービス追加、新端末の対応、既存バグ修正などで含まれる。
この数年の間、E-mailの使い方が大幅に変わったので、色々な調整、対応を行っている。
プロジェクトのコアメンバーとして、ほとんどのサブプロジェクトを参加し、東京と上海に合わせて平均20人のチームで働いてる。\\
主な役割はTechリーダーとBSEである。
Techリーダーとして、技術面はもちろん、最大10人のチームを管理し、C, C++, Perlのプロジェクトは開発してきた。
BSEとして、日本側のメンバーと仕様の相談、技術調査、仕様書作成、中国側とのコミュニケーションをやりました。言語、文化による違いがなるべく吸収してきた。\\
\\
\underline{主なサブプロジェクト}
}
\cventry{}{ネットワーク構造可視化}{}{}{}{
  \begin{itemize}
  \item 本番環境にある数千台サーバーの可視化、経路検索を行うシステム。一から作るシステムであり、新しい技術群を導入しています。本システムの導入により、商用作業の障害数は半分以下に減りました。今もニーズを応えてどんどん新機能を追加しています。
  \item アーキテクチャとして要求分析からシステム全体の設計、サーバー側の実装を行いました。
  \item Perl, NoSql, Restful interface, javascript, html5, 自動化ツール
  \end{itemize}
}
\cventry{}{通信データ統計ツール}{}{}{}{
  \begin{itemize}
  \item 通信データ統計ツール。パフォーマンスのため、複数回全体のアーキテクチャを変わったことがあった。最終的に簡単なの分散式処理フレームワークになった。
  \item 小さいプロジェクトなので、IT試験以外全部やりました。
  \item Perl, 手作り分散式フレームワーク
  \end{itemize}
}
\cventry{}{Linuxへのポーティング}{}{}{}{
  \begin{itemize}
  \item 古いhp/uxで動くcプログラムは今現在のlinuxにポーティングするプロジェクト。OS変更とともに、ハードウェア、ソフトウェア両方のアーキテクチャに変更があったので、ソースコードの改修もあった。
  \item Techリーダーとして、仕様のやり取り、技術調査、仕事の振り分けとコードレビューをやりました。
  \item c, linux, hp/ux, ポーティング
  \end{itemize}
}
\cventry{}{IMAPキャッシュ}{}{}{}{
  \begin{itemize}
  \item 既存の古いimapサーバーの前にキャッシュサーバーを構える。特定のimapコマンドで10倍近く早くなった効果が得た。
  \item プログラマーとして、imapプロトコルのアナライザーとコードレビューをやりました。
  \item C++, boost, asio, async, アジャイル
  \end{itemize}
}
\cventry{}{Zimbraからのメールデータ移行}{}{}{}{
  \begin{itemize}
  \item Zimbraから既存のメールサーバーへのデータ移行デーモン サービス。移行自体だけではなく、あらゆるエラー状況でリカバリできるのもポイントである。
  \item Techリーダーとして、全体のアーキテクチャ、メインロジックのコーディング、仕様のやり取りとコードレビューまでやりました。
  \item C++11, git, zimbra, ldap, mysql, アジャイル, 自動化テスト
  \end{itemize}
}
\cventry{}{WEBメール}{}{}{}{
  \begin{itemize}
  \item 既存メールシステムにWEBメールの機能を提供するプロジェクト。一から開発ではなく、製品を使っていろいろカスタマイズをした。特に既存システムとのインターフェースとUI関連。
  \item Techリーダーとして、技術調査、製品選定、アーキテクチャ設計、既存システムとのインターフェース設計、開発方式選定、仕様のやり取り、とコードレビューをやりました。
  \item postfix, dovecot, mysql, php, nginx, php-fpm, openid
  \end{itemize}
}
\cventry{}{ログ処理システム}{}{}{}{
  \begin{itemize}
  \item プロキシを対応するログ処理システム。ログ処理の中心として、インターフェースの統一、特殊イベントの監視、重要情報の統計、リアルタイム収集 を実装した。
  \item チームリーダーとして、仕様のやり取り、仕事の振り分け、スケジュールの調整、コードレビューと一部のコーディングをやりました。
  \item Perl, 大規模ログ処理
  \end{itemize}
}
\cventry{}{IMAPプロキシ}{}{}{}{
  \begin{itemize}
  \item 性能とメンテナンスの易さを考慮して、既存のimapサーバーにプロキシ レイヤーを追加する。プロキシは本体、情報検索するためのdirectory、送信するためのmds、三つの部分で構成してる。
  \item プログラマーとして、プロキシ内部のコマンドの解析と文字コード変換をやりました。
  \item c, socket, スレッドプール, コードページ
  \end{itemize}
\bigskip
\underline{その他}
}
\cventry{}{システム管理のためのスクリプト}{}{}{}{
  \begin{itemize}
  \item TAT時間測定、ldapアップデート、サービス監視、ログ監視、ログ抽出など
  \item Perl, 自動化
  \end{itemize}
}
\cventry{}{運用とリリース}{}{}{}{
  \begin{itemize}
  \item 日本チームとともに作ったプログラムをお客さん環境にリリース。
  \item リリース手順書、リカバリー手順書などを作る。実際のリリースをする。
  \item ヒューマンミスは最小限にするリリースプロセス
  \end{itemize}
}
\cventry{}{サーバー管理}{}{}{}{
  \begin{itemize}
  \item プロジェクトで使われる複数のサーバーを管理する
  \item OSインストール、アップデート、バックアップなど
  \item hp/ux, linux, virtual machines, version control system, trac, backup
  \end{itemize}
}

\subsection{@フロムソフトウェア}
\cventry{2010/05 - 2010/12}{PSPアクションゲーム}{}{}{}{
モンハン日記 ぽかぽかアイルー村という作品です。
フィッシング パーツ、一部のタイトルと共通ライブラリを作ってた。\\
開発ツールはPSP SDK (C++)、完全な3Dゲームである。
}
\cventry{2008/04 - 2009/05}{DSアドベンチャーゲーム}{}{}{}{
金田一耕助シーリスをDSでゲーム化した初作品。
ミニゲームのクロスワード、クイズ、新聞とゲームタイトルと共通ライブラリを作ってた。\\
開発ツールはDS SDK (C++)、ほとんど2Dゲーム、一部(タイトルなど)は3Dを使ってた。
このゲームはファミ通5点をもらった。
}
\cventry{2007/10 - 2010/12}{ゲーム開発ツールのメンテナンス}{}{}{}{
これは単独なプロジェクトではなく、ほかのゲームの開発とともに、一連の開発ツールをメンテナンスしてた。3D モデリング、リソース管理、ゲーム イベント管理など。
企画の要望とおり機能を追加し、バグ修正などをやりました。\\
主にC++とVisual Basicを使ってた。
}
\subsection{@啓明ソフトウェア}
\cventry{2007/05 - 2007/09}{文書管理システム}{}{}{}{
独自のシステムで、企業で使われるすべての文書は中央文書管理サーバーに保存し、
権限がもつ人だけ見る、編集する事ができる。\\
C++ と mfc を使って、新しいドキュメントフォマットの対応をやりました。
}
\cventry{2006/08 - 2007/05}{文書暗号化システム}{}{}{}{
中央認証機能が持つ文書暗号化システム。
このシステムは認証サーバーとドキュメント フォーマット毎のプラグインでできてるので、
C++, Visual Basic, vba, java, tomcat, jsp など複数な技術を使われる。\\
短時間でこのシステムを把握し、コアメンバーとして新しい暗号化アルゴリズムの追加や
新ドキュメント フォーマットの対応なのをやりました。
}
\cventry{2005/06 - 2006/08}{住宅公団用管理システム}{}{}{}{
元々日本住宅公団用の管理システムで、リニューアルのため C/S モデルから B/S
モデルへ変わるとの要望がありました。\\
結局は元々の Visual Basic で作ったシステムは Java ベースの web 系システムに移植した。
このプロジェクトは大成功であり、お客さんからの表彰を受けた。
}
\cventry{2004/11 - 2005/06}{農場管理システム}{}{}{}{
農場管理用システム。\\
主にJava, Tomcat と Oracleを使ってた。
}
\cventry{2004/09 - 2004/11}{銀行システムインフラ}{}{}{}{
Windows NT 3.5 で動くC++で作った古い銀行システム。\\
インフラの仕様変更で改修を行いました。改修後は2倍近く性能が得てきた。
}
\cventry{2004/07 - 2004/09}{通信プロトコル テスト ツール}{}{}{}{
簡単なネットワーク通信プロトコルを試験するためのツール、初プロジェクトでもある。\\
主にC++, socketとマルチスレッドを使ってた。
}

\section{その他}
\cvlistitem{
数百ユーザ向けの個人E-mailサービスは提供している。
}
\cvlistitem{
十年以上に渡り個人のポケモン関連掲示板を運営していた。
掲示板の構築から内容までほとんど一人でやっていた。
当時の登録ユーザー数は二十万までありました。
}

\nocite{*}
\bibliographystyle{plain}
\bibliography{publications}

\clearpage
\end{spacing}
\end{document}
