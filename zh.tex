%% start of file `template-zh.tex'.
%% Copyright 2006-2012 Xavier Danaux (xdanaux@gmail.com).
%
% This work may be distributed and/or modified under the
% conditions of the LaTeX Project Public License version 1.3c,
% available at http://www.latex-project.org/lppl/.


\documentclass[11pt,a4paper,sans]{moderncv}   % possible options include font size ('10pt', '11pt' and '12pt'), paper size ('a4paper', 'letterpaper', 'a5paper', 'legalpaper', 'executivepaper' and 'landscape') and font family ('sans' and 'roman')

% moderncv 主题
\moderncvstyle{classic}                       % 选项参数是 ‘casual’, ‘classic’, ‘oldstyle’ 和 ’banking’
\moderncvcolor{blue}                          % 选项参数是 ‘blue’ (默认)、‘orange’、‘green’、‘red’、‘purple’ 和 ‘grey’
%\nopagenumbers{}                             % 消除注释以取消自动页码生成功能

% 字符编码
\usepackage[utf8]{inputenc}                   % 替换你正在使用的编码
\usepackage{CJKutf8}

% 调整页面出血
\usepackage[scale=0.8]{geometry}
\recomputelengths                             % required when changes are made to page layout lengths
%\setlength{\hintscolumnwidth}{3cm}           % 如果你希望改变日期栏的宽度

% 个人信息
\firstname{隽杰}
\familyname{奚}
\title{个人简历}                    % 可选项、如不需要可删除本行
%\address{XXXXXXXXXXXXXXXXXXXXXXXXXXXXXXXXXXXXXX}{XXXXXXXXXXXXXXXXXXXXXX}      % 可选项、如不需要可删除本行
%\mobile{XXXXXXXXXXXXXXXXXXX}                  % 可选项、如不需要可删除本行
\address{XXXXXXXXXXXXXXXXXXXXXXXXXXXXXXXXXXXXXXXXXXXXXXXXXXXXXXXXXXXXX}{XXXXXXXXXXXXXXXXXXXXXXX}      % 可选项、如不需要可删除本行
\mobile{XXXXXXXXXXXXXXXXXXX}                  % 可选项、如不需要可删除本行
%\phone{XXXXXXXXXXXXXXXXXX}                   % 可选项、如不需要可删除本行
%\fax{XXXXXXXXXXXXXXXX}                       % 可选项、如不需要可删除本行
\email{XXXXXXXXXXXXXXXXXXXX}                  % 可选项、如不需要可删除本行
%\homepage{XXXXXXXXXXXXX}                      % 可选项、如不需要可删除本行
%\extrainfo{附加信息 (可选项)}                % 可选项、如不需要可删除本行
\photo[64pt][0.4pt]{picture}                  % ‘64pt’是图片必须压缩至的高度、‘0.4pt‘是图片边框的宽度 (如不需要可调节至0pt)、’picture‘ 是图片文件的名字;可选项、如不需要可删除本行
\quote{Summary\newline{}
\small
\begin{itemize}
\item 10年IT行业从业经验。常年海外工作经验。
\item 精通日语,熟练使用英语。
\item 12人规模的管理经验。
\item 精通c/c++, Perl, linux系统管理等技术。
\item 对邮件系统有深入的理解。
\item 日本游戏行业从业经验。
\item 优秀的团队合作精神。
\end{itemize}
\normalsize
}                         % 可选项、如不需要可删除本行

% 显示索引号;仅用于在简历中使用了引言
%\makeatletter
%\renewcommand*{\bibliographyitemlabel}{\@biblabel{\arabic{enumiv}}}
%\makeatother

% 分类索引
%\usepackage{multibib}
%\newcites{book,misc}{{Books},{Others}}
%----------------------------------------------------------------------------------
%            内容
%----------------------------------------------------------------------------------
\begin{document}
\begin{CJK}{UTF8}{gbsn}                       % 详情参阅CJK文件包
\maketitle

\section{信息}
\cvline{生日}{\small 1982年8月28日 (上海)\normalsize}
\cvline{国籍}{\small 中国\normalsize}
\cvline{LinkedIn}{\small \url{http://cn.linkedin.com/in/xijuanjie}\normalsize}
%\cvline{Blog}{\small \url{<Link to profile>}\normalsize}
%\cvline{Skype}{\small <skype>\normalsize}

\section{工作经验}
\cventry{2011年1月 - 至今}{技术主管}{上海惠普有限公司}{东京/上海}{}{}
\cventry{2007年10月 - 2010年12月}{游戏程序员}{From Software}{东京}{日本}{}
\cventry{2004年7月 - 2007年8月}{软件工程师}{上海启明软件有限公司}{上海}{中国}{}

\section{IT技能}
\cvcomputer{语言}{c, c++, Perl, common lisp, php}{数据库}{Sqlite, Mysql, MariaDB}
\cvcomputer{平台}{Debian, Redhat, Gentoo, HP-UX, Mac os x, Windows, Apache, Nginx}{工具}{vim, eclpise, clang, make, git, Mercurial, SVN, *nix binutils}

\section{教育经历}
\cventry{2000 - 2004}{建筑环境与设备工程}{学士}{上海理工大学}{上海}{}  % 第3到第6编码可留白
\cventry{1996 - 2000}{上海建设中学}{}{}{上海}{}  % 第3到第6编码可留白

%\cventry{年 -- 年}{学位}{院校}{城市}{\textit{成绩}}{说明}

%\section{毕业论文}
%\cvitem{题目}{\emph{题目}}
%\cvitem{导师}{导师}
%\cvitem{说明}{\small 论文简介}

\clearpage

%Section
\section{语言能力}

\hspace{25mm}\small 从 A 到 E 的自我评价 (A最高)\normalsize
\vspace{5mm}

\begin{tabular}{p{67mm} p{40mm} p{40mm} p{20mm}}
& \textbf{理解} & \textbf{口语} & \textbf{写作} \\
\end{tabular}

\begin{tabular}{p{67mm} p{20mm} p{20mm} p{20mm} p{20mm} p{20mm}}
& 听力 & 阅读 & 交流 & 工作 & \\
\end{tabular}

\vspace{3mm}
%lvl should be in this range A1 < A2 < B1 < B2 < C1 < C2
\cvlanguage{中文}{母语}{
	\begin{tabular}{p{20mm} p{20mm} p{20mm} p{20mm} p{21mm}}
		A & A & A & A & A
	\end{tabular}}
\cvlanguage{日语}{精通}{
	\begin{tabular}{p{20mm} p{20mm} p{20mm} p{20mm} p{21mm}}
		A & A & A & B & B
	\end{tabular}}
\cvlanguage{英语}{熟练}{
	\begin{tabular}{p{20mm} p{20mm} p{20mm} p{20mm} p{21mm}}
		B & B & C & C & B
	\end{tabular}}

\section{证书}
\cvitemwithcomment{LPIC Level1}{}{2009年12月}
\cvitemwithcomment{LPIC Level2}{}{2013年3月}
\cvitemwithcomment{高级程序员}{}{2006年10月}
\cvitemwithcomment{JLPT 1}{}{2007年8月}
%\cvline{LPIC Level1}{December 2009}
%\cvline{LPIC Level2}{March 2013}
%\cvline{Senior Programmer}{October 2006}
%\cvline{JLPT 1}{August 2007}

\section{考试分数}
\cvline{TOEIC}{765}

\section{项目经验}
\subsection{@惠普有限公司}
\cventry{2011年1月 - 至今}{大规模邮件系统维护}{}{}{}{
这个项目是为日本三大电信公司之一维护一个大规模的手机邮件系统。\newline
维护本身包含加入新的功能,扩展新的服务,支持新的客户端,修改现有的bug等等。
由于客户的市场不断扩张,并且加上近年来手机的邮件客户端的迅速发展(从传统手机渐渐转为智能手机)需要做很多工作来给最终用户提供更好的服务。所以这个项目也一直持续下去。
作为项目中的核心成员,这些年来我和50人左右的团队一起,参加了几乎所有的开发子项目,从中国到日本。
项目中我主要担任技术Leader和Bridge SE的职务。
作为技术LEader, 我领导一个10人的小组,主要使用 c 和 Perl 进行业务扩展的开发。
作为Bridge SE。我和日本的同事们一起讨论系统需求,调查技术难点,写下系统设计,并且和在中国的开发团队进行沟通,以减少文化差异带来的成本。
经过了这3年来的努力工作,整个邮件系统渐渐从一个古老并使用非标准协议的封闭系统成长为一个以开源软件为基础的可灵活扩展的邮件系统。
}
\subsection{@From software}
\cventry{2010年3月 - 2010年12月}{PSP游戏开发}{}{}{}{
这是一个基于怪物猎人系列的休闲动作游戏。
在一个20人的开发团队里,我主要完成了钓鱼的场景和其他一些共通函数。
游戏使用 PSP SDK (c++) 编写,是一个纯 3D 游戏。
}
\cventry{2008年4月 - 2009年5月}{NDS游戏开发}{}{}{}{
这是一个基于著名侦探小说金田一耕助系列的冒险游戏。
我完成了这个游戏的小游戏部分,包括填字游戏,空格计算,新闻浏览,一些共通函数和游戏从标题到进入游戏之前的部分。
游戏主要使用任天堂SDK (c++) 写的,是一个 2D 游戏,但是使用了部分 3D 技术。
这个游戏得到了日本杂志 Famitsu 的五星奖。
}
\cventry{2007年10月 - 2010年12月}{维护游戏开发工具}{}{}{}{
这不是一个独立的项目,是对一系列内部使用的游戏开发工具进行维护。这些工具涉及3D建模,资源管理,和游戏数据制作等。
根据策划和美工的需求,为工具加入新功能,并且修复现有bug。
这些工具使用了不同的开发语言,主要是c++,还有一部分visual basic。
}
\subsection{@启明软件有限公司}
\cventry{2007年5月 - 2007年9月}{文档管理系统}{}{}{}{
这是一个帮助公司以更安全的方式管理文档的系统。
文档会被注册到一个中央服务器上,只有符合权限的人才可以查看和修改。
我使用 c++ 和 mfc 扩展了文件查看器支持的文件格式。
}
\cventry{2006年8月 - 2007年5月}{文档加密系统}{}{}{}{
这是一个使用网络中央认证的文档加密系统。
系统由一个中央认证服务器,和一系列的 Office 和 Adoba Acrobat reader 用的插件组成。
因此系统中使用到了各种技术,包括 c++, visual basic, vba, java, tomcat, jsp 等等。
经过一段时间后,我完全熟悉了这个系统,成为了这个系统的核心开发者,重写了更安全的加密以及加上了更多的文档种类支持。
}
\cventry{2005年6月 - 2006年8月}{住宅管理系统}{}{}{}{
这是一个原本用 visual basic 写的,日本政府用来管理公租房的系统。
根据客户需求,需要把该系统从 C/S 架构改写成 B/S 架构。
试验了几种方案之后,最终决定把原来的 VB 系统改写成 Java 的 web 系统。
这是一个大型项目,最多的时候一共有60个成员。
最终这个项目得到了很大的成功,也受到了客户的年度表彰。
}
\cventry{2004年11月 - 2005年6月}{农业管理系统}{}{}{}{
一个用来帮助农民用可视化的方式管理农作物的系统。
系统使用了 Java, Tomcat 和 Oracle 等技术。
}
\cventry{2004年9月 - 2004年11月}{银行系统}{}{}{}{
一个c++写的非常古老的银行用系统,运行在 Windows NT 3.5 上。
因为业务扩展,需要改善原有代码的运行效率,修改了一部分底层函数。
在完成这部分函数修改之后,系统有了约2倍的速度提高。
}
\cventry{2004年7月 - 2004年9月}{通讯协议测试工具}{}{}{}{
一个十分简单的网络通讯协议测工具,也是我第一个项目。
工具中使用到了c++, socket, 多线程等知识。
}

\section{兴趣爱好}
\cvlistitem{自行车}
\cvlistitem{音乐}

\section{附加信息}
\cvlistitem{
从服务器的基础到内容,全面管理一个论坛超过10年。在最热门的时候,注册用户超过10万。
}

%\section{其他 1}
%\cvlistitem{项目 1}
%\cvlistitem{项目 2}
%\cvlistitem{项目 3}

%\renewcommand{\listitemsymbol}{-}             % 改变列表符号
%
%\section{其他 2}
%\cvlistdoubleitem{项目 1}{项目 4}
%\cvlistdoubleitem{项目 2}{项目 5\cite{book1}}
%\cvlistdoubleitem{项目 3}{}

% 来自BibTeX文件但不使用multibib包的出版物
%\renewcommand*{\bibliographyitemlabel}{\@biblabel{\arabic{enumiv}}}% BibTeX的数字标签
\nocite{*}
\bibliographystyle{plain}
\bibliography{publications}                    % 'publications' 是BibTeX文件的文件名

% 来自BibTeX文件并使用multibib包的出版物
%\section{出版物}
%\nocitebook{book1,book2}
%\bibliographystylebook{plain}
%\bibliographybook{publications}               % 'publications' 是BibTeX文件的文件名
%\nocitemisc{misc1,misc2,misc3}
%\bibliographystylemisc{plain}
%\bibliographymisc{publications}               % 'publications' 是BibTeX文件的文件名

%\clearpage
%%-----       letter       ---------------------------------------------------------
%% recipient data
%\recipient{Company Recruitment team}{Company, Inc.\\123 somestreet\\some city}
%\date{January 01, 1984}
%\opening{Dear Sir or Madam,}
%\closing{Yours faithfully,}
%\enclosure[Attached]{curriculum vit\ae{}}     % use an optional argument to use a string other than "Enclosure", or redefine \enclname
%\makelettertitle
%
%Lorem ipsum dolor sit amet, consectetur adipiscing elit. Duis ullamcorper neque sit amet lectus facilisis sed luctus nisl iaculis. Vivamus at neque arcu, sed tempor quam. Curabitur pharetra tincidunt tincidunt. Morbi volutpat feugiat mauris, quis tempor neque vehicula volutpat. Duis tristique justo vel massa fermentum accumsan. Mauris ante elit, feugiat vestibulum tempor eget, eleifend ac ipsum. Donec scelerisque lobortis ipsum eu vestibulum. Pellentesque vel massa at felis accumsan rhoncus.
%
%Suspendisse commodo, massa eu congue tincidunt, elit mauris pellentesque orci, cursus tempor odio nisl euismod augue. Aliquam adipiscing nibh ut odio sodales et pulvinar tortor laoreet. Mauris a accumsan ligula. Class aptent taciti sociosqu ad litora torquent per conubia nostra, per inceptos himenaeos. Suspendisse vulputate sem vehicula ipsum varius nec tempus dui dapibus. Phasellus et est urna, ut auctor erat. Sed tincidunt odio id odio aliquam mattis. Donec sapien nulla, feugiat eget adipiscing sit amet, lacinia ut dolor. Phasellus tincidunt, leo a fringilla consectetur, felis diam aliquam urna, vitae aliquet lectus orci nec velit. Vivamus dapibus varius blandit.
%
%Duis sit amet magna ante, at sodales diam. Aenean consectetur porta risus et sagittis. Ut interdum, enim varius pellentesque tincidunt, magna libero sodales tortor, ut fermentum nunc metus a ante. Vivamus odio leo, tincidunt eu luctus ut, sollicitudin sit amet metus. Nunc sed orci lectus. Ut sodales magna sed velit volutpat sit amet pulvinar diam venenatis.
%
%Albert Einstein discovered that $e=mc^2$ in 1905.
%
%\[ e=\lim_{n \to \infty} \left(1+\frac{1}{n}\right)^n \]
%
%\makeletterclosing


\clearpage\end{CJK}
\end{document}


%% 文件结尾 `template-zh.tex'.
